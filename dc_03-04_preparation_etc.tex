%form: dc_form_03-04.tex ; user: dc_03-04_preparation_etc.tex
%========== DC =========
%===== p. 03-04 現在までの研究状況 =============
\section{現在までの研究状況}
%watermark: w03_past_dc
\newcommand{\研究の背景}{%
%begin  研究の背景===================
	象の卵の研究の背景は...

	\begin{thebibliography}{99}
		\bibitem{teramura} 寺村輝夫、「ぼくは王様 - ぞうのたまごのたまごやき」.
	\end{thebibliography}
%end  研究の背景 ====================
}

\newcommand{\現在までの研究状況}{%
%begin  現在までの研究状況===================
	\subsection{これまでの研究の背景}
	●\textbf{これまでの研究の背景}

	SNSの発展により、情報を迅速に大量取得し、拡散することで容易に共有できるようになった。
	その一方、悪意によって他人を騙すために作られた\textbf{フェイクニュース}が拡散されやすくなった。
	フェイクニュースが拡散されると、\textbf{誤った認識が広がって騙された人々が社会的損害を起こす}という問題がある。
	たとえば、2016年米国大統領選挙前にフェイクニュースに騙された人々がピザ屋で銃撃事件を起こした\cite{agencies_2016}。
	また、今年は特にCOVID-19にまつわるフェイクニュースが広く拡散され、不安に陥った人々が買いだめを行うことが世界的に問題となった。
	WHOは情報の過剰な氾濫を``インフォデミック''と定義し、テドロス事務局長は\textbf{誤った情報はウイルス以上に拡散されやすい}と指摘した\cite{ZAROCOSTAS2020676}。

	\subsection{問題点}
	●\textbf{問題点}

	現在フェイクニュース対策として有識者が事実関係を確認する\textbf{ファクトチェック}が行われている。
	ただしこれは\textbf{属人的な作業}であり、拡散されてから調査されることが多く、結果を公表するまで時間がかかることからフェイクニュースと比べあまり\textbf{拡散されにくい}。
	また自動で検出する場合、フェイクニュースは事実を模した形をとるため単純な\textbf{ルールベース手法では限界}がある。
	近年ではニューステキストや添付メディア、ユーザの反応を入力にもつ\textbf{ディープニューラルネットワーク(DNN)}を利用した手法がみられる。
	%この場合はブラックボックス問題により\textbf{説明可能性が不足}するため、SNS利用者から支持を得にくい。
	その中で\textbf{ユーザの反応は拡散後でしか得られない}ため、早期発見を想定した場合評価対象にすることができない。

	\subsection{解決方策}
	●\textbf{解決方策}

	そこで、当研究では\textbf{学習でのみユーザの反応を活用}し、テスト時は\textbf{ユーザの反応を生成し補完}して、
	\textbf{精度を落とさず早期発見を目指す}ことにした。

	\subsection{研究目的・研究方法}
	●\textbf{研究目的・研究方法}

	フェイクニュース早期発見に向け、SNS上で\textbf{ニュースに寄せられたコメントを生成する}ことが、
	\textbf{真偽を分類する精度の向上につながる}ことを示す。
	本研究はニュースと寄せられたコメントを、
	ニュース本文と実際にSNS上で投稿されたコメント3件を1ユニットとして扱うことにした。
	
	\subsection{特色と独創的な点}
	●\textbf{特色と独創的な点}
	
	\begin{itemize}
		\item 生成タスクを分類タスクに役立てる点
		\item 精度を失わずに速報性をもつことができる点
		\item 生成されたコメントは説明可能性にも繋げられる点
	\end{itemize}


	\subsection{これまでの研究経過及び得られた結果}
	●\textbf{これまでの研究経過及び得られた結果}


	\bibliography{myreferences}
	\bibliographystyle{junsrt}
	%ぞうの卵はおいしいぞう。
ぞうの卵はおいしいぞう。
ぞうの卵はおいしいぞう。
ぞうの卵はおいしいぞう。
ぞうの卵はおいしいぞう。
ぞうの卵はおいしいぞう。
ぞうの卵はおいしいぞう。
ぞうの卵はおいしいぞう。
ぞうの卵はおいしいぞう。
ぞうの卵はおいしいぞう。
ぞうの卵はおいしいぞう。
ぞうの卵はおいしいぞう。
ぞうの卵はおいしいぞう。
ぞうの卵はおいしいぞう。
ぞうの卵はおいしいぞう。
ぞうの卵はおいしいぞう。
ぞうの卵はおいしいぞう。
ぞうの卵はおいしいぞう。
ぞうの卵はおいしいぞう。
ぞうの卵はおいしいぞう。
ぞうの卵はおいしいぞう。
ぞうの卵はおいしいぞう。
ぞうの卵はおいしいぞう。
ぞうの卵はおいしいぞう。
ぞうの卵はおいしいぞう。
ぞうの卵はおいしいぞう。
ぞうの卵はおいしいぞう。
ぞうの卵はおいしいぞう。
ぞうの卵はおいしいぞう。
ぞうの卵はおいしいぞう。
ぞうの卵はおいしいぞう。
ぞうの卵はおいしいぞう。
ぞうの卵はおいしいぞう。
ぞうの卵はおいしいぞう。
ぞうの卵はおいしいぞう。
ぞうの卵はおいしいぞう。
ぞうの卵はおいしいぞう。
ぞうの卵はおいしいぞう。
ぞうの卵はおいしいぞう。
ぞうの卵はおいしいぞう。
ぞうの卵はおいしいぞう。
ぞうの卵はおいしいぞう。
ぞうの卵はおいしいぞう。
ぞうの卵はおいしいぞう。
ぞうの卵はおいしいぞう。
ぞうの卵はおいしいぞう。
ぞうの卵はおいしいぞう。
ぞうの卵はおいしいぞう。
ぞうの卵はおいしいぞう。
ぞうの卵はおいしいぞう。
ぞうの卵はおいしいぞう。
ぞうの卵はおいしいぞう。
ぞうの卵はおいしいぞう。
ぞうの卵はおいしいぞう。
ぞうの卵はおいしいぞう。
ぞうの卵はおいしいぞう。
ぞうの卵はおいしいぞう。
ぞうの卵はおいしいぞう。
ぞうの卵はおいしいぞう。
ぞうの卵はおいしいぞう。
ぞうの卵はおいしいぞう。
ぞうの卵はおいしいぞう。
ぞうの卵はおいしいぞう。
ぞうの卵はおいしいぞう。
ぞうの卵はおいしいぞう。
ぞうの卵はおいしいぞう。
ぞうの卵はおいしいぞう。
ぞうの卵はおいしいぞう。
ぞうの卵はおいしいぞう。
ぞうの卵はおいしいぞう。
ぞうの卵はおいしいぞう。
ぞうの卵はおいしいぞう。
ぞうの卵はおいしいぞう。
ぞうの卵はおいしいぞう。
ぞうの卵はおいしいぞう。
ぞうの卵はおいしいぞう。
ぞうの卵はおいしいぞう。
ぞうの卵はおいしいぞう。
ぞうの卵はおいしいぞう。
ぞうの卵はおいしいぞう。
ぞうの卵はおいしいぞう。
ぞうの卵はおいしいぞう。
ぞうの卵はおいしいぞう。
ぞうの卵はおいしいぞう。
ぞうの卵はおいしいぞう。
ぞうの卵はおいしいぞう。
ぞうの卵はおいしいぞう。
ぞうの卵はおいしいぞう。
ぞうの卵はおいしいぞう。
ぞうの卵はおいしいぞう。
ぞうの卵はおいしいぞう。
ぞうの卵はおいしいぞう。
ぞうの卵はおいしいぞう。
ぞうの卵はおいしいぞう。
ぞうの卵はおいしいぞう。
ぞうの卵はおいしいぞう。
ぞうの卵はおいしいぞう。
ぞうの卵はおいしいぞう。
ぞうの卵はおいしいぞう。
ぞうの卵はおいしいぞう。
ぞうの卵はおいしいぞう。
ぞうの卵はおいしいぞう。
ぞうの卵はおいしいぞう。
ぞうの卵はおいしいぞう。
ぞうの卵はおいしいぞう。
ぞうの卵はおいしいぞう。
ぞうの卵はおいしいぞう。
ぞうの卵はおいしいぞう。
ぞうの卵はおいしいぞう。
ぞうの卵はおいしいぞう。
ぞうの卵はおいしいぞう。
ぞうの卵はおいしいぞう。
ぞうの卵はおいしいぞう。
ぞうの卵はおいしいぞう。
ぞうの卵はおいしいぞう。
ぞうの卵はおいしいぞう。
ぞうの卵はおいしいぞう。
ぞうの卵はおいしいぞう。
ぞうの卵はおいしいぞう。
ぞうの卵はおいしいぞう。
ぞうの卵はおいしいぞう。
ぞうの卵はおいしいぞう。
ぞうの卵はおいしいぞう。
ぞうの卵はおいしいぞう。
ぞうの卵はおいしいぞう。
ぞうの卵はおいしいぞう。
ぞうの卵はおいしいぞう。
ぞうの卵はおいしいぞう。
ぞうの卵はおいしいぞう。
ぞうの卵はおいしいぞう。
ぞうの卵はおいしいぞう。
ぞうの卵はおいしいぞう。
ぞうの卵はおいしいぞう。
ぞうの卵はおいしいぞう。
ぞうの卵はおいしいぞう。
ぞうの卵はおいしいぞう。
ぞうの卵はおいしいぞう。
ぞうの卵はおいしいぞう。
ぞうの卵はおいしいぞう。
ぞうの卵はおいしいぞう。
ぞうの卵はおいしいぞう。
ぞうの卵はおいしいぞう。
ぞうの卵はおいしいぞう。
ぞうの卵はおいしいぞう。
ぞうの卵はおいしいぞう。
ぞうの卵はおいしいぞう。
ぞうの卵はおいしいぞう。
ぞうの卵はおいしいぞう。
ぞうの卵はおいしいぞう。
ぞうの卵はおいしいぞう。
ぞうの卵はおいしいぞう。
ぞうの卵はおいしいぞう。
ぞうの卵はおいしいぞう。
ぞうの卵はおいしいぞう。
ぞうの卵はおいしいぞう。
ぞうの卵はおいしいぞう。
ぞうの卵はおいしいぞう。
  % << only for demonstration. Please delete it or comment it out.	
%end  現在までの研究状況 ====================
}

