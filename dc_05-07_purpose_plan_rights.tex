%form: dc_form_05-07.tex ; user: dc_05-07_purpose_plan_rights.tex
%========== DC =========
%===== p. 05-07 研究の目的・内容、特色、計画、人権 =============
\subsection{研究の目的・内容}
%watermark: w02_purpose_plan_ugly_dc
\newcommand{\研究目的}{%
%begin  研究目的と内容===================

	\setlength\intextsep{0pt}
	\setlength\textfloatsep{0pt}
	\begin{wrapfigure}{r}[7pt]{0.31\linewidth}
		\centering
		\includegraphics[width=0.31\textwidth]{figs/objects.pdf}
		\vspace{-10mm} 
		\caption{本研究のモデル図}
		\label{fig:objects}
	\end{wrapfigure}

	\vspace{-5mm} 
	\noindent
	●\textbf{研究目的}



	図2の通り、フェイクニュース拡散抑制に向けた説明可能な真偽早期検出手法の開発のため、
	(1)\textbf{生成コメントを使用した分類でF値が}良好な値とされる\textbf{0.8以上の手法の確立}と、
	(2)\textbf{フェイクニュースの真偽の根拠を説明する手法の確立する手法の確立}を行う。

	\noindent
	●\textbf{研究方法・研究内容}


	\noindent
	(1) 生成コメントを使用した分類でF値が0.8を上回る手法の確立: 

	
	\textbf{誤りや見逃しなくフェイクニュースを発見するモデル}を構築する。
	DNNはデータ量の増加によって性能が上がるため、ユーザ情報など、⼊⼒情報を追加する。
	%データセットや入力情報の追加(ユーザ情報等)とそれに伴うモデルの変更で実現を模索する。
	%また、性能向上に向く大規模データセットがない場合は条件に符合するものを自作する。
	%最低でもF値0.7以上を目指す。
	また、同様の手段でモデルの汎用性を向上する。

	\noindent
	(2)フェイクニュースの真偽の根拠を説明する手法の確立:

	Fakeである指摘に向け、\textbf{生成コメントを理由付けへ活用}する。
	また、\textbf{理由付けによるユーザからの信用の変化}を主観実験で評価する。
	RTやいいね等と⽐べ、コメントは⽂章として多くの情報を提供する。
	本研究は拡散初期でコメントが少ないニュースが対象のため、
	⽣成コメントを⽤いて根拠の⾔語化に取り組む。
	%コメントはRTやいいね等と比べ文章として多くの情報を提供する一方、
	%これまで同様早期発見を行う場合数が少ないため生成コメントを活用する。
	%生成コメントから理由付けが難しいならば、実投稿からの実現を目指す。
	⽇本語を対象に⾏う場合は、⾔語や国⺠性による差異も検証項⽬とする。

	\noindent
	●\textbf{所属研究室との関連}

	%所属研究室はエージェント、知的Web、ソフトウェア工学、データマイニングの4つの分野に渡っており、
	%本研究はデータマイニングの一環となる。
	%デマや噂の検出を本研究が扱う対象に含めても
	本研究は所属研究室内でも萌芽的な取り組みで、環境・技術面の支援を除き申請者が全部分を担当する。

	\noindent
	%●\textbf{研究計画の期間中に異なった研究機関(外国の研究機関等を含む。)において研究に従事することを予定}
	●\textbf{研究計画の期間中に異なった研究機関において研究に従事}
	
	申請者は\textbf{期間中1年間タリン工科大学の言語技術研究所(Tanel Alumäe所長)にて活動予定}である。
	当該研究は北米と欧州で研究が活発で(次項目詳説)、申請者は海外における活動実績(9p.詳説)が多いため、
	\textbf{最前線の研究に従事する}ためにも申請者が現地で研究に従事することが必要である。

%end  研究目的と内容 ====================
}

\newcommand{\人権の保護及び法令等の遵守への対応}{%
%begin  人権の保護及び法令等の遵守への対応 ===================
	コメント取得を予定してしているSNSはTwitterである。
	Twitter社は2020年3月より学術目的でTwitter APIの利用を自由化しているほか、
	取得したツイートIDを含む情報をデータセットとして公開することも学術目的であれば認められている\cite{twitter_2020}。

	また、先行研究が提供したデータセットを使用する場合は、提供者が示しているライセンスやポリシーを遵守する。

	なお、学習済みモデルの公表は平成30年改正著作権法第30条4号により認められている。

	ただし、本研究では主観評価実験としてSNSユーザを対象としたアンケート調査を予定している。
	この調査により収集したデータは、個⼈の特定につながる情報を匿名化した上で解析を⾏い、
	解析結果の公表に際しては、匿名化を⾏ったデータを⽤い、個⼈情報の漏洩防⽌に配慮する。

	{\footnotesize
		\begin{thebibliography}{99}
			\setcounter{enumiv}{11}
			\bibitem{twitter_2020} Twitter開発者ポリシーを分かりやすくアップデート, 2020年3月11日. (最終閲覧日 2020年4月19日) \url{https://blog.twitter.com/developer/ja_jp/topics/tools/2020/DevPolicyUpdate.html}
		\end{thebibliography}
	}
%end  人権の保護及び法令等の遵守への対応 ====================
}

\newcommand{\研究の特色と独創的な点tillNextPage}{%
%begin  研究の特色と独創的な点===================
	\vspace*{-1mm}	
	\noindent
	●\textbf{これまでの先行研究等があれば、それらと比較して、本研究の特色、着眼点、独創的な点}

	%先行研究で既に学術的に実現されている事項と、本研究との差異を記述する。

	本研究の新規性は\textbf{フェイクニュース拡散を抑制}するため\textbf{根拠言語化モデルと早期発見モデルとを統合}し、
	\textbf{システム化}することである。

	早期発見に向けユーザの反応を補完する弱教師あり学習を採用したMWSSは既に今年提案された\cite{Shu2020LeveragingMW}。
	本研究はユーザの反応から、ニュースに対するユーザのコメントに注⽬し、根拠の⾔語化の材料に採⽤した。
	ニュースに寄せられそうなコメントを生成する手法は、
	確率分布に従う潜在変数と、正解ラベルから頻出単語を生成するTCNN-URGが提案された\cite{ijcai2018-533}。
	本研究は、真偽判断の根拠を⾔語化するモデルをフェイクニュースの早期発⾒モデルと統合し、
	フェイクニュース拡散抑制システムを構築する。

	フェイクニュース自動検出に説明を提供する手法として、
	記事とコメントから真偽判断の決め手である部分を評価するdEFENDが提案されている\cite{shu2019defend}。
	これは投稿されたコメントが対象で、まだコメントが少ない早期検出に向かない。
	当研究では、\textbf{生成されたコメントから説明可能性の抽出}によって、
	早期検出と両立すると同時に、\textbf{説明付加DNNモデルの応⽤可能性}を⽰す。
	
	\noindent
	●\textbf{国内外の関連する研究の中での当該研究の位置づけ、意義}

	この研究は、ここ\textbf{数年で世界的に競争が激化}した一方、
	その\textbf{研究対象が英語に集中}している。
	%更に本研究は早期検出に加え、上記研究の\textbf{知見のローカライズ}も視野に入れている。
	例えば、Google Scholar上で2015年に投稿された中で``fake news''でヒットする論文は520本に対して、
	同じ条件で2019年に投稿された論文は15,400本と\textbf{4年で約30倍に増加}した。
	%前項目の通り、この研究分野では頻繁に英語論文が発表されている。
	一方、同じ論文プラットフォームで2019年で``フェイクニュース''でヒットする日本語論文は169本と、\textbf{英語の90分の1}にとどまる。
	これは%地域による問題意識の差の他にも、近年DNNを活用した研究が多く、
	DNNに必要な記事と真偽データ等を含む\textbf{大規模データセットが英語に集中}する点が考えられる。
	%フェイクニュース検出なら
	ラベル付けではファクトチェック結果を活用可能\cite{wang-2017-liar}だが、
	北米・欧州に比べ国内のファクトチェックは発展途上であるため、日本語はデータセット作りから着手する必要がある。
	また同様の理由によって、\textbf{言語や国民性による差異まで言及した研究はみられない}。
	%具体的には、ユーザへの拡散抑制を呼びかける場合、言語に問わず同じ方法が有効か否か不明である。
	%「リフレインさせず飽きさせない」ことと「一貫性をもたせる」ことの両立が難しい……
	日本国内のユーザを対象に拡散抑制を大きく促すアプローチ方法の検討のためには、この詳細な差異を明らかにすることが必要とみられる。
	
	当該研究は⾃然な文章の⽣成や真偽分類が⽬的である故、システム提供のような応⽤は考慮されない。
	本研究はフェイクニュース拡散抑制システムを通し、\textbf{⾔語化モデルの応⽤可能性を⽰す}点で意義がある。

	\noindent
	●\textbf{本研究が完成したとき予想されるインパクト及び将来の見通し}
	
	%本研究が完成すると、SNSのユーザへ情報が事実かフェイクかを判断する新しい判断材料を早い段階からもたらすことができる。
	%また、フェイクニュースであることを\textbf{早い段階で説得力がある理由も併せて指摘}することにより、
	%ユーザによる\textbf{拡散を抑制}することができる。
	%古今東西で虚偽情報を流布する人々は存在するが、
	本研究の完成で\textbf{ユーザが簡単に騙されないための新たな判断材料}により、
	\textbf{ジャーナリズムと民主主義に対する最大の脅威であるフェイクニュース}\cite{zhou2019wsdm}から\textbf{ユーザを守る}ことが可能となる。
	更に言語を跨ぎこのシステムを提供することができれば、地域を選ばずこの知見を共有することができる。
	
	さらにフェイクニュース検出以外でも、
	前述の言語や国民性やニュースの分野による傾向差は\textbf{心理学上でも大きな知見}である。
	更に情報発信者に対して\textbf{誤情報を発信する前に警告するシステム}など、
	拡散のみならず発信でもこのモデルは応用ができる。
	現代社会にて、ユーザ一人ひとりの情報の発信や拡散に大きな責任が伴うが、その認識は広まっていない。
	%ユーザにその認識が広まっているとは言い難い。
	本研究はこの状況へ極めて⼤きなインパクトがある。

	
	%\vspace*{-1mm}

	{\footnotesize
		\begin{thebibliography}{99}
			\setcounter{enumiv}{7}
			\vspace*{-2mm}
			\setlength{\parskip}{0cm}
			\setlength{\itemsep}{0cm}
			\begin{spacing}{0.7}
				\bibitem{ijcai2018-533} Feng Qian, \textit{et al.} Neural user response generator: Fake news de-tection with collective user intelligence. In \textit{Proc. of the IJCAI-18}, pp. 3834–3840., 2018.
				\bibitem{Shu2020LeveragingMW} Kai Shu, \textit{et al.} Leveraging multi-source weak social supervision for early detection of fake news. \textit{arXiv}, Vol.abs/2004.01732, 2020.
				\bibitem{wang-2017-liar}William Yang Wang. ``Liar, Liar Pants on Fire'': A New Benchmark Dataset for Fake News Detection. In \textit{Proc. of the 55th Annual Meeting of the Association for Computational Linguistics (Volume 2: Short Papers)}, pp.422-426, 2017.
				\bibitem{zhou2019wsdm}Xinyi Zhou, et al. Fake news: Fundamental theories, detection strategies and challenges. In \textit{Proc. of the WSDM '19}, pp. 836–837, 2019.
			\end{spacing}
		\end{thebibliography}
		%\bibliography{myreferences}
		%\bibliographystyle{junsrt}
	}
%end  研究の特色と独創的な点 ====================
}

\newcommand{\研究計画withVspaceDC}{%
%begin  研究計画 ===================
	% 今年の様式は困ったものです。ブツブツ。
	% 今の技術では「(4)研究計画」のスタート位置を指定できることができません。
	% すみませんが、次の2行を調整してください。
%	\clearpage	% もし「(3) 研究の特色・独創的な点」がp.6に流れこまないなら、この行を有効にしてください。
	\vspace*{12mm}	% 「(4) 研究計画」が正しい高さから始まるように、この値を調整してください。

	%\textbf{この行の高さは、上の「研究の特色・独創的な点」との間隔を\textbackslash vspaceを用いて調整してください。}


	%\vspace*{-7mm}

	\begin{wrapfigure}{r}[7pt]{0.63\textwidth}
		\centering
		\vspace*{-5mm}
		\includegraphics[width=0.63\textwidth]{figs/chart.pdf}
		\vspace*{-10mm}
		\caption{本研究の年次計画(1セルは半期を表す)}
		\label{tbl:chart}
	\end{wrapfigure}

	本研究の3年間の流れを図\ref{tbl:chart}に示す。	
	申請時点から採用までの期間は、現有モデルの分類精度を改善や、
	計画AとBに向くモデルとデータセットの選定・作成の戦術立案を行う。

	\noindent
	●\textbf{1年目}

	\noindent
	\textbf{A. データセットの選定・作成}

	本研究が使うデータセットを、論文の調査やGoogle Dataset Search等の検索エンジンで随時選定する。
	条件はタスクによるが、少なくともニュースと真偽、そしてユーザのコメントが必要である。
	満足なデータセットがなければ、業績3-1にて申請者が使用したTwitter API等で取得し、
	再現性の観点より取得したデータセットの内容を公開する。

	\noindent
	\textbf{B. コメント生成・真偽分類モデルの実装}

	コメント生成・真偽分類モデルの実装を引き続き行う。
	生成コメントを含めた真偽分類では、分類の総合指標であるF値0.8を目標とする。
	もしも現有モデルの拡張で実現が難しいならば、ユーザの反応を生成・補完し評価するMWSSなど本研究に近い他手法からの拡張を検討している。
	再現性の確保のため、実装した学習済モデルやそれを使ったデモの公開も予定している。

	%\noindent
	%\textbf{C. 真偽分類性能向上}

	%本研究では生成コメントを含めた真偽判定において、分類の総合指標であるF値が0.8を上回ることを目指している。
	%データセットの規模拡大やパラメータの調整、分類モデルの変更などで実現を目指す。

	\noindent
	●\textbf{2年目}

	\noindent
	\textbf{C. 別言語・ドメインへの対応}

	言語やニュースのトピックであるドメインの変化に提案モデルを対応させる。
	特に日本語に対応する場合、形態素解析や事前学習済み日本語単語の分散表現、そしてデータセットの用意が必要である。
	%また、いずれも同時にデータセットも新たに用意しなければならない。
	もし多言語対応が難しいならば、Bで既に新規性は示されるため英語内での別ドメイン対応を予定している。

	\noindent
	\textbf{D. 言語性・国民性による差異の検討}

	Cにより、言語やそれを使用する国民性によってフェイクニュース拡散傾向に違いがみられるか調べ、具体的な⾔語化モデルの考案や、早期検出モデルへの適⽤に取り組む。
	具体的には原文記事と翻訳記事を対象に分類成績や生成されたコメント同士、そして拡散ネットワークを比較する形で実現する。
	%これが国による具体的なフェイクニュースとの触れ方を明らかにし、より具体的な説明可能性の提供や早期検出への道筋となる。

	\noindent
	\textbf{E. 説明可能性の付与}

	ユーザに説明を提供するため、生成コメントから真偽を判断した材料を取得する。
	これは生成・分類モデルを拡張することによって実現する。
	また、Dの結果によっては出力の形式を変更・調整するほか、オンライン上でのデモの提供も予定している。
	もしも生成されたコメントから説明可能性が得られない場合は、実際に投稿されたコメントと記事から得ることを予定している。

	\noindent
	●\textbf{3年目}

	\noindent
	\textbf{F. 拡散抑制効果の評価}

	実際にSNS上で提供した時を想定し、Bによって分類成績を改善させ、Eによって説明可能性を付与したモデルの効果を測定する。
	具体的には提案モデルがSNSユーザへの拡散意欲やモデルへの信憑性にどのような影響があるか、10点評価によるアンケート調査を用いた主観評価実験で評価する。
	%また、Dに付随しユーザの使用言語によって手法を変えた場合の評価も兼ねる。

%end  研究計画 ====================
}

