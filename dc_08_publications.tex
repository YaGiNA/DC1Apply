%form: dc_form_08.tex ; user: dc_08_publications.tex
%========== DC =========
%===== p. 08 研究業績 =============
\section{研究業績}
%    	\begin{small} %------------------------
\renewcommand{\応募者の研究遂行能力}{%
%begin  研究遂行能力 ===================
	応募者は過去20年間、7つの海を隅から隅まで航海し、
	浅瀬から深海まで潜り、文字通り東西南北上下の3次元で
	シロナガスクジラの卵の探索を行ってきた。
	シロナガスクジラに飲み込まれそうになったり、海賊に捕まるなどの危険な目にも
	あったが、それにもめげず、研究を遂行してきた強靭な能力を有する。
%end  研究遂行能力 ====================
}

\subsection{学術雑誌(紀要・論文集等も含む)に発表した論文及び著書}
\newcommand{\学術雑誌等に発表した論文または著書}{%
%begin  学術雑誌等に発表した論文または著書===================
	
	\begin{enumerate}
		\item[](査読有り)%===========================
		\item \underline{H. Yukawa}$^1$, J. Kara$^2$,
				``Theory of Elephant Eggs'', 
				Phys.\ Rev.\ Lett. {\bf 800}, 800-804 (2005). 
				\label{pub:theoegg}
				
		\item F.~Ehrlich, \underline{H. Yukawa}$^1$,
				``You can't Lay an Egg If You're an Elephant'', 
				JofUR\\
				 ({\tt www.universalrejection.org}), {\bf N/A}, N/A (2002).

		\item[](査読なし)%=============================
		\item Kobo Abe$^3$, \underline{H. Yukawa}$^1$, 
				``仔象は死んだ'', 
				安部公房全集, {\bf 26}, 100-200, (2004).
	\end{enumerate}
	他5報
%end  学術雑誌等に発表した論文または著書 ====================
}

\subsection{学術雑誌等又は商業誌における解説・総説}
\newcommand{\学術雑誌等または商業誌における解説や総説}{%
%begin  学術雑誌等または商業誌における解説や総説===================
	\begin{enumerate}
		\item R.~Kipling, \underline{H. Yukawa},
				``The Elephant's Child (象の鼻はなぜ長い)'', 
				Nature, {\bf 999}, 777-779, (2003).
	\end{enumerate}
	他2件
%end  学術雑誌等または商業誌における解説や総説 ====================
}

\subsection{国際会議における発表}
\newcommand{\国際会議における発表}{%
%begin  国際会議における発表===================
	\begin{enumerate}
		\item $\circ$ 湯川秀樹、
			``Theory of Elephant Eggs'', 
			原始殻物理国際会議、
			カラチ、2006年2月

%		\item $\circ$ 湯川秀樹、Jacques-Yves Cousteau,
%			``How to search for whale eggs'',
%			国際海洋探索会議、ハワイ、2003年4月
	\end{enumerate}
	他1件
%end  国際会議における発表 ====================
}

\subsection{国内学会・シンポジウムにおける発表}
\newcommand{\国内学会やシンポジウムにおける発表}{%
%begin  国内学会やシンポジウムにおける発表===================
	\begin{enumerate}
		\item $\circ$ 湯川秀樹、朝永振一郎、
			「ほ乳類の真の意味」、
			ほ乳類学会、
			東京、2003年6月
	\end{enumerate}
	他3件
%end  国内学会やシンポジウムにおける発表 ====================
}

\subsection{特許等}
\newcommand{\特許等}{%
%begin  特許等===================
	\begin{enumerate}
		\item[](公開中)
		\item 800800号、「クジラの卵を用いた深海潜水艇」\underline{湯川秀樹}、2003年4月
%		\item[] (申請中)
%		\item 8000000号、「象の卵を用いた(ひ・み・つ)」、\underline{湯川秀樹}、2007年4月
	\end{enumerate}		
%end  特許等 ====================
}

\subsection{その他の業績}
\newcommand{\その他の業績}{%
%begin  その他の業績===================
		\begin{enumerate}
			\item もうすぐもらえるで賞
		\end{enumerate}
%end  その他の業績 ====================
}

%	\end{small}
