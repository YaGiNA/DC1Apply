%form: dc_form_08.tex ; user: dc_08_publications.tex
%========== DC =========
%===== p. 08 研究業績 =============
\section{研究業績}
%    	\begin{small} %------------------------
\renewcommand{\応募者の研究遂行能力}{%
%begin  研究遂行能力 ===================
	申請者はこれまで研究室に配属されてからフェイクニュースの自動検出というトピックに取り組み続けている。
	研究発表する際には社会情勢の影響もあってか非常に本研究に対する期待感を発表時によく耳にしている。
	また、自然言語処理技術の急速な発展により、技術的な障壁は年々下がりつつある。
%end  研究遂行能力 ====================
}

\subsection{学術雑誌(紀要・論文集等も含む)に発表した論文及び著書}
\newcommand{\学術雑誌等に発表した論文または著書}{%
%begin  学術雑誌等に発表した論文または著書===================
	
	なし

%end  学術雑誌等に発表した論文または著書 ====================
}

\subsection{学術雑誌等又は商業誌における解説・総説}
\newcommand{\学術雑誌等または商業誌における解説や総説}{%
%begin  学術雑誌等または商業誌における解説や総説===================

	なし
%end  学術雑誌等または商業誌における解説や総説 ====================
}

\subsection{国際会議における発表}
\newcommand{\国際会議における発表}{%
%begin  国際会議における発表===================

	なし
%end  国際会議における発表 ====================
}

\subsection{国内学会・シンポジウムにおける発表}
\newcommand{\国内学会やシンポジウムにおける発表}{%
%begin  国内学会やシンポジウムにおける発表===================

	広学医サイのポスター発表はここに含めていいのか?→高校に問い合わせる	
	
	(本研究との直接の関連が皆無)

	(以下1件 査読なし・口頭発表)
	\begin{enumerate}
		\item \underline{$\circ$ 栁裕太}、田原康之、大須賀昭彦、清雄一\\
			「画像付きフェイクニュースとジョークニュースの検出・分類に向けた機械学習モデルの検討」、\\
			日本ソフトウェア科学会 2018年度 MACC研究発表会、
			大分、2019年3月
	\end{enumerate}
%end  国内学会やシンポジウムにおける発表 ====================
}

\subsection{特許等}
\newcommand{\特許等}{%
%begin  特許等===================

	なし
%end  特許等 ====================
}

\subsection{その他の業績}
\newcommand{\その他の業績}{%
%begin  その他の業績===================

プレプリント論文

7月開催の国際学会INESに投稿中、採録なら(3)に移管予定

一応通知は5月3日らしいけど多分コロナでぶっ飛ぶからキレそう
	\begin{enumerate}
		\item \underline{$\circ$ Yuta Yanagi}, Ryouhei Orihara, Yuichi Sei, Yasuyuki Tahara, and Akihiko Ohsuga.\\
		``Fake news detection with generated comments for news articles''.
		EasyChair Preprint no. 3190, EasyChair, 2020.

	\end{enumerate}
%end  その他の業績 ====================
}

%	\end{small}
