%form: dc_form_08.tex ; user: dc_08_publications.tex
%========== DC =========
%===== p. 08 研究業績 =============
\section{研究業績}
%    	\begin{small} %------------------------
\renewcommand{\応募者の研究遂行能力}{%
%begin  研究遂行能力 ===================
	申請者は2018年度に研究室に配属されてからフェイクニュースの自動検出というトピックに取り組み続け、2019年3月にMACCにて最初の成果発表を行った(業績4-1)。
	また、今年7月にIEEEハンガリー支部が主催するINESへの発表を予定している(業績3-1)。
	%研究発表する際には社会情勢の影響もあってか非常に本研究に対する期待感を発表時によく耳にしている。
	また、自然言語処理技術の急速な発展により、環境に起因する障壁は年々下がりつつある。

	申請者は研究活動の経験を早い段階から積んでおり、高校生の段階で研究実績を挙げている(業績4-2)。

	%業績3-1は採録されたら入れる。リジェクトならプレプリントに。 
%end  研究遂行能力 ====================
}

\subsection{学術雑誌(紀要・論文集等も含む)に発表した論文及び著書}
\newcommand{\学術雑誌等に発表した論文または著書}{%
%begin  学術雑誌等に発表した論文または著書===================
	
	なし

%end  学術雑誌等に発表した論文または著書 ====================
}

\subsection{学術雑誌等又は商業誌における解説・総説}
\newcommand{\学術雑誌等または商業誌における解説や総説}{%
%begin  学術雑誌等または商業誌における解説や総説===================

	なし
%end  学術雑誌等または商業誌における解説や総説 ====================
}

\subsection{国際会議における発表}
\newcommand{\国際会議における発表}{%
%begin  国際会議における発表===================

	(以下1件 査読あり・口頭発表予定)
	\begin{enumerate}
		\item \underline{$\circ$ Yuta Yanagi}, Ryouhei Orihara, Yuichi Sei, Yasuyuki Tahara, and Akihiko Ohsuga.\\
		``Fake news detection with generated comments for news articles''.
		The 24th IEEE International Conference on Intelligent Engineering Systems 2020, (Reykjavík, Iceland)
		Virtual event due to COVID-19, July 2020(Accepted).

	\end{enumerate}

	%なし
%end  国際会議における発表 ====================
}

\subsection{国内学会・シンポジウムにおける発表}
\newcommand{\国内学会やシンポジウムにおける発表}{%
%begin  国内学会やシンポジウムにおける発表===================

	(以下1件 査読なし・口頭発表)
	\begin{enumerate}
		\item \underline{$\circ$ 栁裕太}、田原康之、大須賀昭彦、清雄一\\
			「画像付きフェイクニュースとジョークニュースの検出・分類に向けた機械学習モデルの検討」、\\
			日本ソフトウェア科学会 2018年度 MACC研究発表会、
			大分、2019年3月
	\end{enumerate}

	(以下1件 査読なし・ポスター発表)
	\begin{enumerate}
		\setcounter{enumi}{1}
		\item \underline{$\circ$ 栁裕太}、葛西透麿、 森谷薫平、神谷岳洋、藤原徹、木村健太、榎本裕介\\
			「CaD428の変異遺伝子の機能解析ツールの汎用化」、\\
			広尾学園高校医進・サイエンスコース研究成果報告会、
			東京、2015年3月
	\end{enumerate}
%end  国内学会やシンポジウムにおける発表 ====================
}

\subsection{特許等}
\newcommand{\特許等}{%
%begin  特許等===================

	なし
%end  特許等 ====================
}

\subsection{その他の業績}
\newcommand{\その他の業績}{%
%begin  その他の業績===================

	なし
%プレプリント論文(INES採録なら(3)に移管予定)

	%\begin{enumerate}
	%	\item \underline{$\circ$ Yuta Yanagi}, Ryouhei Orihara, Yuichi Sei, Yasuyuki Tahara, and Akihiko Ohsuga.\\
	%	``Fake news detection with generated comments for news articles''.
	%	EasyChair Preprint no. 3190, EasyChair, 2020.

	%\end{enumerate}
%end  その他の業績 ====================
}

%	\end{small}
