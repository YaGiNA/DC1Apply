%form: dc_form_09.tex ; user: dc_09_myself.tex
%========== DC =========
%===== p. 09 自己評価 =============
\section{自己評価}
\newcommand{\自己評価}{%
%begin  自己評価===================
\noindent
●\textbf{研究職を志望する動機}

\textbf{申請者は嘘が蔓延ることで誰かが謂れのない罪で傷付く社会に大きな問題意識と危機感を抱いている}。
虚偽と指摘されているにも関わらず、誤った認識が改善されない事案が多発し、強いもどかしさを抱いている。
解決するためには、嘘を発信させないことよりも、嘘を拡散させないユーザの意識醸成が重要と申請者は考える。
なぜならば、嘘を流布させようとする人々はどの時代にも存在するためだ。
%SNSの発展によりすぐに拡散されるようになったため、拡散を抑制する策が追いついていないことが問題なのである。

自然言語処理技術の観点から嘘に騙されない社会作りに必要な技術と知見を迅速に提供することができれば、
あるいは利用者がフェイクニュースの拡散を少しでも思いとどまらせることができるかもしれない。
そのためには、申請者は研究者として実現方法を検討することが必要である。

\noindent
●\textbf{目指す研究者像}

申請者が本研究を実現するために必要な研究者像がもつ資質として以下の3つを挙げる。

\vspace*{-3mm}

\begin{enumerate}
    \setlength{\parskip}{0cm}
    \setlength{\itemsep}{0cm}
    \item \textbf{自分が抱える問題意識や目標から今やるべきことまで切り分ける能力}
    \item \textbf{今やるべきことの理由を把握しやり切る覚悟}
    \item \textbf{他者の視点に立って形而上の自分の考えを具体化して説明する配慮}
\end{enumerate}

\vspace*{-3mm}

研究活動は答えなき社会課題の解決法を探索する。
一見途方もないように見えるが、切り分けを進めることにより、今何をするべきか明確にすることができる。
時に自分がわからないことに直面した場合は、他者に説明し助言を仰ぐことも重要である。
その際には他者の視点に立って自分の考えを具体化することで、スムーズな共有が可能となる。

\noindent
●\textbf{自己の長所}

申請者の長所は、現状の問題点を独自に分析し、解決のために主体的な活動を積極的に行う点である。
どのような状況でも改善を第一に考え今必要なことを洗い出し、時には他者を巻き込み物事を推し進めることができる。
次項目より、長所が役立った具体的な事例も含めて申請者のこれまでの活動を記載する。

\noindent
●\textbf{自己評価をする上で、特に重要と思われる事項}

\textbf{(受賞歴・取得資格)}

申請者はプレゼンテーション能力が高く、早稲田大学本位田研究室との合宿におけるグループ発表にて2季連続最優秀発表賞を受賞している。
また、本研究のベースとなるコンピュータ・サイエンス技術を他者に示すため、2018年に基本情報技術者を取得した。

\textbf{(留学経験)}

申請者は中学時代では豪州にて5週間、高校時代はUC Davisにて2.5週間、学部1年時にASUにて4週間の語学留学を行っており、定期的に海外で英語学習を行った。
また、学部3年時にインドネシアのバンドン工科大学にて現地大学研究室に滞在しスマートシティ構想に関する研究活動を40日間に渡り行った。
ここでも自分でロードマップを作成し成果発表まで英語で活動した。
このように、国内に限らず海外においても英語によるコミュニケーション能力を高める活動に積極的に取り組んでいる。

\textbf{(特色ある学外活動)}

申請者は大学入学直後に\textbf{プログラミングの講義がないことに危機感}を覚え、自ら2つの行動を起こした。

1つ目は\textbf{大学主催の小〜高校生向けプログラミング教室の立ち上げへの参画及び講師活動}である。
実際に教える言語(Python)の習得を目的とした輪講に積極的に参加し、
講師として開講から2年弱にわたり毎週子供たちのプログラミング活動のメンタリングを行った。
このときの経験が、\textbf{他者の視点に立った説明が必要だと強く認識するようになるきっかけ}となった。

もう1つは\textbf{エンジニア活動の開始}である。
高校時代の経験を話し、自分を長期インターンシップとして採用してくれる企業を探すことにした。
合計\textbf{20社に連絡をとり}、アメリエフ株式会社にて1年半エンジニア活動を行った。
また、その後は株式会社フィックスターズにて2.5週、
株式会社justInCase Technologiesにて1年半以上にわたって活動するなど、
\textbf{精力的に産業界でも自らの技術を磨く}ようにしている。

%end  自己評価 ====================
}

% hook9 : right before \end{document} ============
