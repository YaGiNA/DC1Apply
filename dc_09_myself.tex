%form: dc_form_09.tex ; user: dc_09_myself.tex
%========== DC =========
%===== p. 09 自己評価 =============
\section{自己評価}
\newcommand{\自己評価}{%
%begin  自己評価===================
\noindent
●\textbf{研究職を志望する動機}

\textbf{申請者は嘘が蔓延ることで誰かが謂れのない罪で傷付く社会に大きな問題意識と危機感を抱いている}。
解決するためには、嘘を発信させないことよりも、嘘を拡散させないユーザの意識醸成が重要と考える。
なぜならば、プロパガンダを含めると嘘を流布させようとする人々はどの時代にも存在するためだ。
SNSの発展により情報がすぐ拡散されるようになったため、実害が発生するまでの時間が大幅に早まっていることが問題なのである。
自然言語処理技術の観点から嘘に騙されない社会作りに必要な技術と知見を迅速に提供することができれば、
あるいは利用者がフェイクニュースの拡散を少しでも思いとどまらせることができるかもしれない。
そのためには、申請者は研究者として実現方法を検討することが必要である。

\noindent
●\textbf{目指す研究者像}

申請者が本研究を実現するために必要な研究者像がもつ資質として以下の3つを挙げる。

\vspace*{-3mm}

\begin{enumerate}
    \setlength{\parskip}{0cm}
    \setlength{\itemsep}{0cm}
    \item \textbf{自分が抱える問題意識や目標から今やるべきことまで切り分ける能力}
    \item \textbf{今やるべきことの理由を把握しやり切る覚悟}
    \item \textbf{他者の視点に立って形而上の自分の考えを具体化して説明する配慮}
\end{enumerate}

\vspace*{-3mm}

問題意識のみでは、行動を起こせない。解消には何が必要か、長中短期的な目標が解決への道筋を照らす。
それに対して今やるべきことが今後何に繋がるのか、明確な理由がやり切るモチベーションとなる。
そして自分の考えを他者と共有する場合は、相手のバックグラウンドを予測した内容にすることで、忌憚なく実りある議論にすることができる。
申請者は、日々この3点を常に意識し研究活動に臨んでいる。

\noindent
●\textbf{自己の長所}

申請者は能動的に自ら目標達成へ働きかけ、自分で立てた計画通りに遂行できる所と考える。
これは高校時代の研究活動から既にこの長所が強く出た。
申請者がかつて所属していた広尾学園高校の医進・サイエンスコースは、毎年3月に研究成果報告会を行う。
所属大学の前期試験の合格発表後、申請者は2週間で研究成果を出し発表を行うことにした。
自ら目的達成に最も適したプログラミング言語の選定から独学で文法を習得し、実装作業からポスターの作成まで期限まで立てたスケジュール通り遂行した。
%前期試験で進路を確定させた生徒が研究成果報告会で発表を行ったのは申請者が初である。

\noindent
●\textbf{自己評価をする上で、特に重要と思われる事項}

\textbf{(受賞歴・取得資格)}

申請者はプレゼンテーション能力が高く、早稲田大学本位田研究室との合宿におけるグループ発表にて2季連続最優秀発表賞を受賞している。
また、本研究のベースとなるコンピュータ・サイエンス技術を示すため、2018年に基本情報技術者を取得した。

\textbf{(留学経験)}

申請者は中学時代では豪州にて5週間、高校時代はUC Davisにて2.5週間、学部1年時にASUにて4週間の語学留学を行っており、定期的に海外で英語学習を行った。
また、学部3年時にインドネシアのバンドン工科大学にて現地大学研究室に滞在しスマートシティ構想に関する研究活動を40日間に渡り行った。
ここでも自分でロードマップを作成し成果発表まで英語で活動した。

\textbf{(特色ある学外活動)}

申請者は大学入学直後にプログラミングの講義がないことに危機感を覚え、自ら2つの行動を起こした。

1つ目は大学主催の小〜高校生向けプログラミング教室の立ち上げへの参画及び講師活動である。
実際に教える言語(Python)の習得を目的とした輪講に積極的に参加し、
講師として開講から2年弱にわたり毎週子供たちのプログラミング活動のメンタリングを行った。

もう1つはエンジニア活動の開始である。
高校時代の経験を話し、自分を長期インターンシップとして採用してくれる企業を探すことにした。
合計20社に連絡をとり、アメリエフ株式会社にて1年半エンジニア活動を行った。
また、その後は株式会社フィックスターズにて2.5週、
株式会社justInCase Technologiesにて1年半以上にわたって活動するなど、
精力的に産業界でも自らの技術を磨くようにしている。

%end  自己評価 ====================
}

% hook9 : right before \end{document} ============
