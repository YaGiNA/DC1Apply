\documentclass[11pt,a4paper,uplatex,twoside,dvipdfmx]{ujarticle} 	% for uplatex
%\documentclass[11pt,a4paper,twoside,dvipdfmx]{jarticle}		% platex
%==== 科研費LaTeX =============================================
%	2021(R03)年度 DC
%============================================================
% 2008-03-08: Taku Yamanaka (JSPS Research Center for Science Systems / Osaka Univ.)
% 2009-03-03: Yoko Yanagida (Assistant)
% 2010-03-03: Taku: Imported new features introduced in 2009 fall.
% 2011-02-25: Taku: Revised for JFY2013.
% 2013-03-14: Taku: Revised for JFY2014.
% 2014-02-22: Taku: Revised for JFY2015.
% 2016-02-26: Taku: Revised for JFY2017.
%============================================================
\input{forms/form00_header}
% user01_header
%=== 様式のファイルの形式の指定 =================
%   PDFではなく、eps の様式を読み込む場合は、次の行の頭に「%」をつけてください。
\setboolean{usePDFform}{true}
%===================================
\input{forms/form01_header}

% user02_header
%=== 予算の表の印刷 =====================
% 予算の集計の表を出すためには、次の行の頭の%を消してください。
%\setboolean{BudgetSummary}{true}
%=================================

%=== For English, uncomment the next line to left-justify inside table columns.
%\renewcommand{\KLCLLang}{\KLCL}

% === 一部のページだけタイプセット ==============
% New in 2009 fall version!
% 選んだページだけタイプセットするには、次の例の頭の%を消し、並べてください。
% 複数のページを選ぶこともできます。
% 提出前には、必ず全てコメントアウト(頭に%をつける)してください。
%ーーーーーーーーーーーーーーーーーーーーーーーーーーーーーーーーー
%\KLTypesetPage{1}			% p.1 (or p.1を含む連続したページ),
%\KLTypesetPage{3}			% p.3 (or p.3を含む連続したページ),
%\KLTypesetPagesInRange{5}{6}	% p.5 ~ p.6,
%\KLTypesetPagesInRange{8}{10}	% and p.8 ~ p.10
%=================================

% ===== my favorite packages ====================================
% ここに、自分の使いたいパッケージを宣言して下さい。
\usepackage{wrapfig}
\usepackage{url}
\usepackage{setspace}
\usepackage{wrapfig}
% \usepackage{amssymb}
%\usepackage{mb}
% \usepackage{color} % でも科研費の書類はグレースケールで印刷されます
%\DeclareGraphicsRule{.tif}{png}{.png}{`convert #1 `dirname #1`/`basename #1 .tif`.png}
%==========================================================

\newcommand{\KLShouKeiLine}[1]{\cline{#1}}
%もし、小計の上の線を取れと事務に言われたら、
%「そのようなことは、記入要項に書かれていないし、学振はそのようなことは気にしていない。」と
% 突っぱねる。
% それでもなお消せと理不尽なことを言われたら、次の行の 最初の「%」を消す。	
%\renewcommand{\KLShouKeiLine}[1]{}

\newcommand{\KLBudgetTableFontSize}{small}	% 予算の表のフォントの大きさ: small, footnotesize
\newcommand{\KLFundsTableFontSize}{small}	%応募中、受入れ予定の研究費のフォントの大きさ:normalsize, small, footnotesize

% ===== my personal definitions ==================================
% ここに、自分のよく使う記号などを定義して下さい。
\newcommand{\klpionn}{K_L \to \pi^0 \nu \overline{\nu}}
\newcommand{\kppipnn}{K^+ \to \pi^+ \nu \overline{\nu}}


\input{forms/hook3} % for future maintenance
% ===== Global definitions for the PD form ======================
% 基本情報
%
%------ 研究課題名  -------------------------------------------
\newcommand{\研究課題名}{SNS上のフェイクニュースを早期検出するDNNモデルの構築}

%----- 研究機関名と研究代表者の氏名-----------------------
\newcommand{\研究機関名}{電気通信大学}
\newcommand{\申請者氏名}{栁 裕太}
\newcommand{\研究代表者氏名}{\申請者氏名}

%---- 研究期間の最終年度 ----------------
\newcommand{\研究期間の最終元号年度}{34}	%平成で、半角数字のみ
%=========================================================
% ===== Global year-dependent definitions for the Kakenhi form ===========
% 基本情報
\newcommand{\研究開始年度}{2021}
\newcommand{\研究開始元号年度}{03}	%令和

\newcommand{\一年目西暦}{2021}
\newcommand{\二年目西暦}{2022}
\newcommand{\三年目西暦}{2023}
\newcommand{\四年目西暦}{2024}
\newcommand{\五年目西暦}{2025}
\newcommand{\六年目西暦}{2026}

\newcommand{\一年目}{3}
\newcommand{\二年目}{4}
\newcommand{\三年目}{5}
\newcommand{\四年目}{6}
\newcommand{\五年目}{7}
\newcommand{\六年目}{8}

\newcommand{\一年目J}{3}
\newcommand{\二年目J}{4}
\newcommand{\三年目J}{5}
\newcommand{\四年目J}{6}
\newcommand{\五年目J}{7}
\newcommand{\六年目J}{8}


%<<< 
\input{forms/form03_header}
\input{forms/form05_dc_header}
%==========================================================
% form07_header.tex
%	2009-03-04: Taku Yamanaka (Osaka Univ.)
%	2014-03-02: Taku: Moved graphics part to form01_header.tex.
%	2015-08-26: Taku: Added a test for \KLFirstPageIsLongPage.
%==========================================================
% Remember Standard Positions that were set in form05_xxxx_header.tex
\let \KLStandardOddMultiPageLeftEdge = \KLOddMultiPageLeftEdge
\let \KLStandardOddMultiPageRightEdge = \KLOddMultiPageRightEdge
\let \KLStandardEvenMultiPageLeftEdge = \KLEvenMultiPageLeftEdge
\let \KLStandardEvenMultiPageRightEdge = \KLEvenMultiPageRightEdge

\let \KLStandardMultiPageTopEdge = \KLMultiPageTopEdge
\let \KLStandardMultiPageBottomEdge = \KLMultiPageBottomEdge

\let \KLStandardOddLeftEdge = \KLOddLeftEdge
\let \KLStandardOddRightEdge = \KLOddRightEdge
\let \KLStandardEvenLeftEdge = \KLEvenLeftEdge
\let \KLStandardEvenRightEdge = \KLEvenRightEdge

%------ This should be set before \begin{document} ------
\KLStandardLengths
\KLStandardPositions

\ifthenelse{\boolean{KLFirstPageIsLongPage}}{%
	\setlength{\textheight}{10000pt}%
}{%
}
%----------------------------------------------------------------------------


%============================================================
%endPrelude

\begin{document}
% hook5 : right after \begin{document} ==============
\newcommand{\応募者の研究遂行能力}{}		% patch 2020-04-05
 % for future maintenance
%============================================================
%     User Inputs
%============================================================
%form: dc_form_03-04.tex ; user: dc_03-04_preparation_etc.tex
%========== DC =========
%===== p. 03-04 現在までの研究状況 =============
\section{現在までの研究状況}
%watermark: w03_past_dc
\newcommand{\研究の背景}{%
%begin  研究の背景===================
	象の卵の研究の背景は...

	%\begin{thebibliography}{99}
	%	\bibitem{teramura} 寺村輝夫、「ぼくは王様 - ぞうのたまごのたまごやき」.
	%\end{thebibliography}
	%\bibliographystyle{junsrt}
%end  研究の背景 ====================
}

\newcommand{\現在までの研究状況}{%
%begin  現在までの研究状況===================
	\subsection{これまでの研究の背景}
	●\textbf{これまでの研究の背景}

	SNSの発展により、情報を迅速に大量取得し、拡散することで容易に共有できるようになった。
	その一方、悪意によって他人を騙すために作られた\textbf{フェイクニュース}が拡散されやすくなった。
	フェイクニュースが拡散されると、\textbf{誤った認識が広がって騙された人々が社会的損害を起こす}という問題がある。
	たとえば、2016年米国大統領選挙前にフェイクニュースに騙された人々がピザ屋で銃撃事件を起こした\cite{agencies_2016}。
	また、今年は特にCOVID-19にまつわるフェイクニュースが広く拡散され、不安に陥った人々が買いだめを行うことが世界的に問題となった。
	WHOは情報の過剰な氾濫を``インフォデミック''と定義し、テドロス事務局長は\textbf{誤った情報はウイルス以上に拡散されやすい}と指摘した\cite{ZAROCOSTAS2020676}。

	\subsection{問題点}
	●\textbf{問題点}

	現在フェイクニュース対策として有識者が事実関係を確認する\textbf{ファクトチェック}が行われている。
	ただしこれは\textbf{属人的な作業}であり、拡散されてから調査されることが多く、結果を公表するまで時間がかかることからフェイクニュースと比べあまり\textbf{拡散されにくい}。
	また自動で検出する場合、フェイクニュースは巧妙に実際のニュースを模した形をとるため単純な\textbf{ルールベース手法では難しい}。
	近年ではニューステキストや添付メディア、ユーザの反応を入力にもつ\textbf{ディープニューラルネットワーク(DNN)}を利用した手法がみられる。
	%この場合はブラックボックス問題により\textbf{説明可能性が不足}するため、SNS利用者から支持を得にくい。
	その中で\textbf{ユーザの反応は拡散後でしか得られない}ため、早期発見を想定した場合評価対象にすることができない。

	\subsection{解決方策}
	●\textbf{解決方策}

	そこで、当研究では\textbf{学習でのみユーザの反応を活用}し、テスト時は\textbf{ユーザの反応を生成し補完}して、
	\textbf{精度を落とさず早期発見を目指す}ことにした。

	\subsection{研究目的・研究方法}
	●\textbf{研究目的・研究方法}

	フェイクニュース早期発見に向け、SNS上で\textbf{ニュースに寄せられたコメントを生成する}ことが、
	\textbf{真偽を分類する精度の向上につながる}ことを示す。
	本研究はニュースと寄せられたコメントを、
	ニュース本文と実際にSNS上で投稿されたコメント3件を1ユニットとして扱うことにした。
	
	\begin{figure}[ht]
		\centering
		\includegraphics[width=0.95\linewidth]{model.pdf}
		\caption{分類タスクの流れ。コメント生成モデルで1件コメントを生成し真偽分類に活用する。}
		\label{fig:model}
	\end{figure}

	\subsection{特色と独創的な点}
	●\textbf{特色と独創的な点}
	
	\begin{itemize}
		\item 生成タスクを分類タスクに役立てる点
		\item 精度を失わずに速報性をもつことができる点
		\item 生成されたコメントは説明可能性にも役立てられる点
	\end{itemize}

	\subsection{これまでの研究経過及び得られた結果}
	●\textbf{これまでの研究経過及び得られた結果}

	申請者はデータセットとしてFakeNewsNet\cite{Shu2018FakeNewsNetAD}を使用した。
	これはファクトチェックによって\textbf{真偽が評価済である英文ニュース}と、それに\textbf{Twitter上で言及された投稿(ツイート)}等をもつ。
	当研究では最低3件以上英文でコメントとしてツイートが寄せられたニュースを真偽で各2000件使用した。
	拡散の初期段階ではコメントの数は期待できないため、使用するコメントは各3件ずつ無作為に選出し、残りは対象から除外した。

	生成・分類モデルはフェイクニュースを自動で作成するGroverモデル\cite{NIPS2019_9106}を拡張する形で実装した。
	このモデルはフェイクニュースをドメイン・著者・投稿日・見出し・本文の5要素に分け、\textbf{ランダムで歯抜けにして予測させる形で生成学習を実現}したものである。
	今回はこれをユニットの4要素(記事本文と3件のコメント)での実装を目指し調整を行った。
	訓練が完了したコメント生成モデルを使い、図\ref{fig:model}のように\textbf{コメントを1件歯抜けにさせたユニットに生成コメントを付加した上でRealかFakeか分類させた}。
	分類モデルはGroverが提供したものを流用した。

	その結果、生成コメントを含めた場合の\textbf{再現率(Fake記事を見抜いた割合)が0.79}と、歯抜けのまま分類させたときの\textbf{0.75}とコメントなしで分類したときの\textbf{0.62}を上回る結果を得た。
	つまり、コメントを生成することで\textbf{ファクトチェックが必要な疑わしい記事をより多く検出}することができた。

	同時に、生成されたコメントで頻出した単語の傾向において真偽で大きな違いはみられなかった。
	これは、投稿された\textbf{コメントのカテゴリによる出現単語傾向の差は軽微}であることを意味した。

	{\small 
	%\bibliography{myreferences}
	%\bibliographystyle{junsrt}
	\begin{thebibliography}{99}
		\bibitem{agencies_2016} Guardian staff and agencies. Washington gunman motivated by fake news `pizzagate' conspiracy,12 2016.
		\bibitem{ZAROCOSTAS2020676} John Zarocostas. How to fight an infodemic. \textit{The Lancet}, Vol. 395, No. 10225, p. 676, 2020.
		\bibitem{Shu2018FakeNewsNetAD} Kai Shu, Deepak Mahudeswaran, Suhang Wang, Dongwon Lee, and Huan Liu. Fakenewsnet: Adata repository with news content, social context and dynamic information for studying fake newson social media. \textit{ArXiv}, Vol. abs/1809.01286, , 2018.
		\bibitem{NIPS2019_9106} Rowan Zellers, Ari Holtzman, Hannah Rashkin, Yonatan Bisk, Ali Farhadi, Franziska Roesner,and Yejin Choi. Defending against neural fake news. In H. Wallach, H. Larochelle, A. Beygelz-imer, F. d'Alch ́e-Buc, E. Fox, and R. Garnett, editors, \textit{Advances in Neural Information Processing Systems 32}, pp. 9054–9065. Curran Associates, Inc., 2019.
	\end{thebibliography}
	}
	%\input{blahblah}  % << only for demonstration. Please delete it or comment it out.	
%end  現在までの研究状況 ====================
}


%form: dc_form_05-07.tex ; user: dc_05-07_purpose_plan_rights.tex
%========== DC =========
%===== p. 05-07 研究の目的・内容、特色、計画、人権 =============
\subsection{研究の目的・内容}
%watermark: w02_purpose_plan_ugly_dc
\newcommand{\研究目的}{%
%begin  研究目的と内容===================
	\noindent
	●\textbf{研究目的}

	本研究では、SNS上でフェイクニュースの拡散を抑制するために、早期自動検出の精度向上と説明可能性の付与を目的とする。
	具体的には、(1)\textbf{生成されたコメントを使用した分類でF値が0.8を上回る手法の確立}と、
	(2)\textbf{生成されたコメントからユーザへ説明可能性を提案する手法の確立}を目指す研究を行う。

	\noindent
	●\textbf{研究方法・研究内容}

	\noindent
	(1) 生成されたコメントを使用した分類でF値が0.8を上回る手法の確立: 

	早期発見できてもユーザから信用が得られない狼少年的なモデルではなく、
	\textbf{的確にフェイクニュースだけを発見させることが可能なモデル}を構築する。
	また、性能向上に向く大規模データセットがない場合は条件に符合するものを自作する。
	最低でもF値0.7以上を目指す。

	\noindent
	(2)生成されたコメントからユーザへ説明可能性を提案する手法の確立:

	SNS上でフェイクニュースの疑いが強い指摘をする場合を想定し、\textbf{生成したコメントを理由付けの題材として活用する}ことを目指す。
	また、\textbf{理由付けの有無によってSNS利用者からの信用を得られる}ことを主観実験で示す。
	生成コメントから理由付けが難しいならば、実投稿からの実現を目指す。

	\noindent
	●\textbf{所属研究室との関連}

	当研究室はエージェント、知的Web、ソフトウェア工学、データマイニングの4つの分野に渡っており、
	本研究はデータマイニングの一環となる。
	また、デマや噂の検出を含めても本研究には前例はなく、\textbf{申請者が当研究室で初めて本研究に着手}した。

	\noindent
	●\textbf{研究計画の期間中に異なった研究機関(外国の研究機関等を含む。)において研究に従事することを予定}
	
	申請者は\textbf{期間中1年間北米あるいは欧州の研究施設での活動を予定}している。
	国内外でフェイクニュース対策研究には温度差が見られ、特に北米と欧州では盛んに研究が行われている(次項目で詳説)ことから、
	\textbf{最前線の研究に従事する}ためにも申請者が現地で研究に従事することが必要である。

%end  研究目的と内容 ====================
}

\newcommand{\人権の保護及び法令等の遵守への対応}{%
%begin  人権の保護及び法令等の遵守への対応 ===================
	象の卵のES細胞の培養、象のクローンの生成などは行わない。
	象個体を現地から持ち出すことはないので、ワシントン条約ならびに
        生物多様性条約に抵触しない。また、組換え実験は行なわないので、
        カルタヘナ議定書にも抵触しない。
%end  人権の保護及び法令等の遵守への対応 ====================
}

\newcommand{\研究の特色と独創的な点tillNextPage}{%
%begin  研究の特色と独創的な点===================
	\noindent
	●\textbf{これまでの先行研究等があれば、それらと比較して、本研究の特色、着眼点、独創的な点}

	ニュースに寄せられそうなコメントを生成する手法は、
	確率分布に従った潜在変数と正解ラベルを使用して頻出単語を生成するTCNN-URGが提案されている\cite{ijcai2018-533}。
	本研究は\textbf{頻出単語を生成するのではなく}、説明可能性に繋ぎやすい\textbf{実際に投稿されたようなコメントを生成する}ことを目指している。

	また、速報性を維持するためにユーザの反応を補完する弱教師あり学習を活用した手法であるMWSSも既に今年提案されている\cite{Shu2020LeveragingMW}。
	本研究では\textbf{生成する対象をコメントに絞っている}。
	他のユーザの反応(リツイート、いいね、反応したユーザ情報)はコメントに比べて説明可能性に繋げにくいためである。

	フェイクニュース対策に説明可能性を提供する手法として、
	記事とコメントから真偽判断の決め手となった部分を評価するdEFENDが提案されている\cite{shu2019defend}。
	これは既に投稿されたコメントを対象に含むため、まだコメントが多く寄せられていない状況である\textbf{早期発見を目指す場合には向かない}。
	当研究では、\textbf{生成されたコメントから説明可能性を提供する}ことで早期発見を実現する。

	\noindent
	●\textbf{国内外の関連する研究の中での当該研究の位置づけ、意義}

	風評やデマの自動検出も当該研究に含めると、歴史は決して浅くない。
	ただし、ここ\textbf{数年で社会情勢の変化によって一気に世界的に競争が激化}した。

	例えば、Google Scholar上で2015年に投稿された中で``fake news''でヒットする論文は\textbf{520本}に対して、
	同じ条件で2019年に投稿された論文は\textbf{15,400本}と実に\textbf{30倍近くに増加}した。

	前項目の通り、特に北米や欧州から頻繁に英語論文が発表されている。
	先述と同じプラットフォームで2019年で``フェイクニュース''でヒットする論文は\textbf{169本}と、1年間で実に日本語の\textbf{90倍以上}の英語論文が発表されている。

	これは\textbf{地域による問題意識の差}の他にも、特に(本研究を含め)近年機械学習やDNNを活用した研究が多いことも考えられる。
	これらの手法に必要な記事と真偽データなどを含む\textbf{大規模データセットが英語に集中している}のである。
	フェイクニュース検出の場合、ファクトチェック結果をラベル付けに流用することができるが、
	北米・欧州に比べて日本国内のファクトチェックは発展途上であるため、日本語データセットが少ない。
	もしも日本語を研究対象に含める場合、まずはデータセット作りから着手する必要がある。

	\noindent
	●\textbf{本研究が完成したとき予想されるインパクト及び将来の見通し}

	本研究が完成すると、SNSの利用者へこの情報が事実かフェイクかを判断する新しい判断材料を早い段階からもたらすことができる。
	また、フェイクニュースがSNS上で\textbf{早い段階で説得力がある理由によって指摘}することにより、
	利用者による\textbf{拡散を抑制}することができる。
	古今東西で虚偽情報を流布しようとする人々は存在するが、\textbf{利用者が簡単に騙されないような仕組み作り}を行うことで、
	\textbf{ジャーナリズムと民主主義に対する最大の脅威であるフェイクニュースから人々を守る}ことが可能となる。
	
	{\small
		\begin{thebibliography}{99}
			\setcounter{enumiv}{6}
			\bibitem{ijcai2018-533} Feng Qian, \textit{et al.} Neural user response generator: Fake news de-tection with collective user intelligence. In \textit{Proc. of the IJCAI-18}, pp. 3834–3840., 2018.
			\bibitem{Shu2020LeveragingMW} Kai Shu, \textit{et al.} Leveraging multi-source weak social supervision for early detection of fake news. \textit{arXiv}, Vol.abs/2004.01732, 2020.
			\bibitem{shu2019defend} Kai Shu, \textit{et al.} defend: Explainable fake news detection. In \textit{Proc. of the ACM SIGKDD}, 2019.
		\end{thebibliography}
		%\bibliography{myreferences}
		%\bibliographystyle{junsrt}
	}
%end  研究の特色と独創的な点 ====================
}

\newcommand{\研究計画withVspaceDC}{%
%begin  研究計画 ===================
	% 今年の様式は困ったものです。ブツブツ。
	% 今の技術では「(4)研究計画」のスタート位置を指定できることができません。
	% すみませんが、次の2行を調整してください。
%	\clearpage	% もし「(3) 研究の特色・独創的な点」がp.6に流れこまないなら、この行を有効にしてください。
	\vspace*{16mm}	% 「(4) 研究計画」が正しい高さから始まるように、この値を調整してください。

	%\textbf{この行の高さは、上の「研究の特色・独創的な点」との間隔を\textbackslash vspaceを用いて調整してください。}

	本研究の3年間のスケジュールを以下の表\ref{tbl:chart}に示す。

	\vspace*{-7mm}

    \begin{table}[h]
		\caption{本研究の年次計画(1セルは半期を表す)}
		\includegraphics[width=\linewidth]{figs/chart.pdf}
		\label{tbl:chart}
    \end{table}
	
	\vspace*{-9mm}

	\noindent
	●\textbf{1年目}

	\noindent
	\textbf{A. データセットの選定・作成}

	以下の各タスクで使用するデータセットを随時選定する。
	条件はタスクによるが、例えばBならニュースとその真偽、そしてユーザのコメントである。
	もしも条件を満たすデータセットがない場合は、データセットを自分で集める必要がある。	

	\noindent
	\textbf{B. コメント生成・真偽分類モデルの実装}

	コメントを生成し分類するモデルの実装を引き続き行う。
	もしも現有モデルの拡張では難しい場合は別の手法からの拡張も検討している。

	\noindent
	\textbf{C. 真偽分類性能向上}

	本研究では生成コメントを含めた真偽判定において、分類の総合指標であるF値が0.8を上回ることを目指している。
	データセットの規模拡大やパラメータの調整、分類モデルの変更などで実現を目指す。

	\noindent
	●\textbf{2年目}

	\noindent
	\textbf{D. 別言語・ドメインへの対応}

	言語やニュースのトピックであるドメインの変動に提案モデルを対応させる。
	特に日本語対応する場合、形態素解析や事前学習済み日本語単語の分散表現の用意が必要である。
	また、いずれも同時にデータセットも新たに用意しなければならない。

	\noindent
	\textbf{E. 説明可能性の付与}

	ユーザに説明可能性を提供するために、生成されたコメントから真偽を判断した材料を取得する。
	これは分類モデルを拡張することによって実現が可能であると考えている。

	\noindent
	●\textbf{3年目}

	\noindent
	\textbf{F. 拡散抑止力の評価}

	実際にSNS上で提供した時を想定し、分類成績を改善させ説明可能性を付与したモデルがSNS利用者への意識にどのような影響を与えるか主観評価実験によって評価する。
	もしも生成されたコメントから説明可能性が得られない場合は、実際に投稿されたコメントや記事から得ることを予定している。


%end  研究計画 ====================
}


%form: dc_form_08.tex ; user: dc_08_publications.tex
%========== DC =========
%===== p. 08 研究業績 =============
\section{研究業績}
%    	\begin{small} %------------------------
\renewcommand{\応募者の研究遂行能力}{%
%begin  研究遂行能力 ===================
	申請者は2018年度に研究室に配属されてからフェイクニュースの自動検出というトピックに取り組み続け、2019年3月にMACCにて最初の成果発表を行った(業績4-1)。
	また、今年7月にIEEEハンガリー支部が主催するINESへの発表を予定している(業績3-1※)。\red{※採録されたら入れる。リジェクトならプレプリントに。} 
	%研究発表する際には社会情勢の影響もあってか非常に本研究に対する期待感を発表時によく耳にしている。
	また、自然言語処理技術の急速な発展により、環境に起因する障壁は年々下がりつつある。

	申請者は研究活動の経験を早い段階から積んでおり、高校生の段階で研究実績を挙げている(業績4-2)。
%end  研究遂行能力 ====================
}

\subsection{学術雑誌(紀要・論文集等も含む)に発表した論文及び著書}
\newcommand{\学術雑誌等に発表した論文または著書}{%
%begin  学術雑誌等に発表した論文または著書===================
	
	なし

%end  学術雑誌等に発表した論文または著書 ====================
}

\subsection{学術雑誌等又は商業誌における解説・総説}
\newcommand{\学術雑誌等または商業誌における解説や総説}{%
%begin  学術雑誌等または商業誌における解説や総説===================

	なし
%end  学術雑誌等または商業誌における解説や総説 ====================
}

\subsection{国際会議における発表}
\newcommand{\国際会議における発表}{%
%begin  国際会議における発表===================

	なし
%end  国際会議における発表 ====================
}

\subsection{国内学会・シンポジウムにおける発表}
\newcommand{\国内学会やシンポジウムにおける発表}{%
%begin  国内学会やシンポジウムにおける発表===================

	(以下1件 査読なし・口頭発表)
	\begin{enumerate}
		\item \underline{$\circ$ 栁裕太}、田原康之、大須賀昭彦、清雄一\\
			「画像付きフェイクニュースとジョークニュースの検出・分類に向けた機械学習モデルの検討」、\\
			日本ソフトウェア科学会 2018年度 MACC研究発表会、
			大分、2019年3月
	\end{enumerate}

	(以下1件 査読なし・ポスター発表)
	\begin{enumerate}
		\setcounter{enumi}{1}
		\item \underline{$\circ$ 栁裕太}、葛西透麿、 森谷薫平、神谷岳洋、藤原徹、木村健太、榎本裕介\\
			「CaD428の変異遺伝子の機能解析ツールの汎用化」、\\
			広尾学園高校医進・サイエンスコース研究成果報告会、
			東京、2015年3月
	\end{enumerate}
%end  国内学会やシンポジウムにおける発表 ====================
}

\subsection{特許等}
\newcommand{\特許等}{%
%begin  特許等===================

	なし
%end  特許等 ====================
}

\subsection{その他の業績}
\newcommand{\その他の業績}{%
%begin  その他の業績===================

プレプリント論文

7月開催の国際学会INESに投稿中、採録なら(3)に移管予定

一応通知は5月4日だけどコロナでオンライン開催になるらしいです、ぶっ飛ばずに済んだ……
	\begin{enumerate}
		\item \underline{$\circ$ Yuta Yanagi}, Ryouhei Orihara, Yuichi Sei, Yasuyuki Tahara, and Akihiko Ohsuga.\\
		``Fake news detection with generated comments for news articles''.
		EasyChair Preprint no. 3190, EasyChair, 2020.

	\end{enumerate}
%end  その他の業績 ====================
}

%	\end{small}

%form: dc_form_09.tex ; user: dc_09_myself.tex
%========== DC =========
%===== p. 09 自己評価 =============
\section{自己評価}
\newcommand{\自己評価}{%
%begin  自己評価===================
{\bf 1. 研究職を志望する動機、目指す研究者像、自己の長所等}

私は、今までの世界観を全く変える発見をしたい。
そのためには、あくせく金を稼ぐ普通の仕事ではなく、
じっくりと研究と野球に取り組める研究職しかない。

私が理想とする研究者は、三四郎の友人でもある、野々宮宗八である。
彼は俗世間の現実に煩わされることなく、
地下の実験室で黙々と光の圧力の測定に取り組んでいる。
彼こそが、真の研究者である。

自分の長所は、まじめで賢いことである。

\vspace{5mm}
{\bf 2. 自己評価をする上で、特に重要と思われる事項}

間もなく、ノーベル賞受賞予定。
%end  自己評価 ====================
}

% hook9 : right before \end{document} ============


%endUserFiles
\input{forms/hook7} % for future maintenance

% dc_forms
%=======================================
\ifthenelse{\boolean{BudgetSummary}\OR\boolean{klTypesetPage0}}{
	\input{forms/coverpage}
}{}

\KLInputIfPageInRangeIsSelected{1}{2}{forms/dc_form_03-04}
\KLInputIfPageInRangeIsSelected{3}{5}{forms/dc_form_05-07}
\KLInputIfSelected{6}{forms/dc_form_08}
\KLInputIfSelected{7}{forms/dc_form_09}


%========================================


%endFormatFile

\input{forms/hook9} % for future maintenance
\end{document}
